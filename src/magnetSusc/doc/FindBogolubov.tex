%\documentclass[prl,aps,twocolumn,floatfix]{revtex4}
%\usepackage{amsmath,graphics,epsfig,color,verbatim,ulem}

\documentclass[twocolumn,prb,preprintnumbers,amsmath,amssymb,floatfix]{revtex4}
\usepackage[linktocpage,bookmarksopen,bookmarksnumbered]{hyperref}
\usepackage{graphicx}
\usepackage{dcolumn}
\usepackage{braket}
\usepackage{bbold}

\usepackage{amsmath,graphics,epsfig,color,verbatim,ulem}
\newcommand{\eps}{\epsilon} \renewcommand{\a}{\alpha}
\renewcommand{\b}{\beta} \newcommand{\vR}{{\mathbf{R}}}
\renewcommand{\vr}{{\mathbf{r}}} \newcommand{\vk}{{\mathbf{k}}}
\newcommand{\vK}{{\mathbf{K}}} \newcommand{\vq}{{\mathbf{q}}}
\newcommand{\vQ}{{\mathbf{Q}}} \newcommand{\vPhi}{{\mathbf{\Phi}}}
\newcommand{\vS}{{\mathbf{S}}}
\newcommand{\cG}{{\cal G}}
\newcommand{\cF}{{\cal F}} \newcommand{\cD}{{\cal D}}
\newcommand{\Tr}{\mathrm{Tr}} \newcommand{\npsi}{\underline{\psi}}
\newcommand{\vA}{{\mathbf{A}}} \newcommand{\vE}{{\mathbf{E}}}
\newcommand{\vj}{{\mathbf{j}}} \newcommand{\vv}{{\mathbf{v}}}
\newcommand{\kb}{k_B} \newcommand{\cellvol}{}
\newcommand{\trace}{\mbox{Tr}} \newcommand{\ra}{\rangle }
\newcommand{\la}{\langle } \newcommand{\om}{\omega}
\newcommand{\up}{\uparrow}
\newcommand{\dn}{\downarrow}
\renewcommand{\Im}{\mathrm{Im}} 
\renewcommand{\Re}{\mathrm{Re}} 
\newcommand{\nphi}{\underline{\phi}}
\newcommand{\tIm}{\overline{\Im}}


\begin{document}
\special{papersize=8.5in,11in}
\setlength{\pdfpageheight}{\paperheight}
\setlength{\pdfpagewidth}{\paperwidth}


\title{How to compute the gap with $\Delta$ in orbital space}
\maketitle

\section{Refresh on Nambu formalism}

Nambu-Gorkov spinor is
\begin{eqnarray}
\Psi^\dagger_{\vk,i}=(c_{\vk i \up}^\dagger, c_{-\vk i \dn})  
\end{eqnarray}
and the Nambu Green's function is
\begin{widetext}
\begin{eqnarray}
G_{\vk,ij}(\tau)=-\langle T_\tau
\left(
\begin{array}{c}
c_{\vk i \up}(\tau)\\
c^\dagger_{-\vk i \dn}(\tau)
\end{array}
\right)
\left(
\begin{array}{cc}
c^\dagger_{\vk j \up} & c_{-\vk j \dn}
\end{array}
\right)\rangle
=-
\left(
\begin{array}{cc}
\langle T_\tau c_{\vk i \up}(\tau) c^\dagger_{\vk j \up}\rangle  &\langle T_\tau c_{\vk i \up}(\tau) c_{-\vk j \dn}\rangle\\
\langle T_\tau c^\dagger_{-\vk i \dn}(\tau) c^\dagger_{\vk j \up}\rangle & \langle T_\tau c^\dagger_{-\vk i \dn}(\tau) c_{-\vk j \dn}\rangle
\end{array}
\right)
\end{eqnarray}
\end{widetext}

We define
\begin{eqnarray}
\cG_{\vk,ij}(\tau) = -\langle T_\tau c_{\vk i \up}(\tau) c^\dagger_{\vk j \up}\rangle\\
\cF_{\vk,ij}(\tau) = -\langle T_\tau c_{\vk i \up}(\tau) c_{-\vk j \dn}\rangle
\end{eqnarray}
and see that
\begin{eqnarray}
G_{22}&=&-\langle T_\tau c^\dagger_{-\vk i \dn}(\tau) c_{-\vk j \dn}\rangle=
\langle T_\tau c_{-\vk j \dn}(-\tau) c^\dagger_{-\vk i \dn} \rangle=
\nonumber\\
&=&-\cG_{-\vk,ji}(-\tau)\nonumber\\
G_{12}&=&-\langle T_\tau c^\dagger_{-\vk i \dn}(\tau) c^\dagger_{\vk j \up}\rangle=
\cF^*_{\vk,ji}(\tau)
\end{eqnarray}

To prove the last identity, we check $\tau>0$ case, and write
\begin{eqnarray}
\cF^*_{\vk,ij}(\tau)=
-\frac{1}{Z}\Tr\left(
e^{-\beta H^*} e^{\tau H^*} c^*_{\vk i \up}e^{-\tau H^*} c^*_{-\vk j \dn}
\right)=\nonumber\\
-\frac{1}{Z}\Tr\left(
\left(e^{-\beta H+\tau H}\right)^T (c^\dagger_{\vk i \up})^T \left(e^{-\tau H}\right)^T (c^\dagger_{-\vk j \dn})^T
\right)=\nonumber\\
-\frac{1}{Z}\Tr\left(
\left(
c^\dagger_{-\vk j \dn}
e^{-\tau H}
c^\dagger_{\vk i \up}
e^{-\beta H+\tau H}
\right)^T
\right)=\nonumber\\
-\frac{1}{Z}\Tr\left(
e^{-\beta H+\tau H}
c^\dagger_{-\vk j \dn}
e^{-\tau H}
c^\dagger_{\vk i \up}
\right)=\nonumber\\
-\langle
T_\tau c^\dagger_{-\vk j \dn}(\tau)c^\dagger_{\vk i \up}
\rangle
\end{eqnarray}
Hence, Bogolubov Green's function is
\begin{eqnarray}
G_{\vk,ij}(\tau)=
\left(
\begin{array}{cc}
\cG_{\vk,ij}(\tau) & \cF_{\vk,ij}(\tau)\\
\cF^*_{\vk,ji}(\tau) & -\cG_{-\vk,ji}(-\tau)
\end{array}
\right)
\end{eqnarray}
or in matrix notation
\begin{eqnarray}
G_{\vk}(\tau)=
\left(
\begin{array}{cc}
\cG_{\vk}(\tau) & \cF_{\vk}(\tau)\\
\cF^\dagger_{\vk}(\tau) & -\cG^{T}_{-\vk}(-\tau)
\end{array}
\right)
\end{eqnarray}
and in frequency
\begin{eqnarray}
G_{\vk}(i\omega)=
\left(
\begin{array}{cc}
\cG_{\vk}(i\omega) & \cF_{\vk}(i\omega)\\
\cF^\dagger_{\vk}(-i\omega) & -\cG^{T}_{-\vk}(-i\omega)
\end{array}
\right)
\end{eqnarray}

\section{ab-initio DMFT and SC gap}
We first write LDA+DMFT solution in eigenbasis, which is frequency
dependent
\begin{eqnarray}
\cG_{\vk}(i\omega,\vr\vr') = \psi^R_{\vk l}(i\omega,\vr)\frac{1}{i\omega+\mu-\varepsilon_{\vk l, i\omega}} \psi^L_{\vk l}(i\omega,\vr')
\end{eqnarray}
or
\begin{eqnarray}
\int d\vr d\vr' \psi^L_{\vk l}(i\omega,\vr) \cG^{-1}_{\vk}(i\omega,\vr\vr') \psi^R_{\vk l}(i\omega,\vr')= {i\omega+\mu-\varepsilon_{\vk l, i\omega}} 
\nonumber
\end{eqnarray}
The DMFT projector $U_{\vr\alpha}$, which is used to embed the
self-energy, can embed gap and give its real space representation as
\begin{eqnarray}
\Delta^\vk(\vr\vr')=U_{\vr\alpha}\Delta^\vk_{\alpha\beta}U^\dagger_{\beta\vr'}  
\end{eqnarray}
being non-local in band index or real space.
We can transform it to DMFT eigenbasis by
\begin{eqnarray}
\bar{\Delta}^\vk_{ll'}(i\omega)=\int d\vr d\vr' \psi^L_{\vk l}(i\omega,\vr) U_{\vr\alpha}\Delta^\vk_{\alpha\beta}U^\dagger_{\beta\vr'}   \psi^R_{\vk l}(i\omega,\vr')
\end{eqnarray}
These projectors are computed by "dmftgk" in "e" mode, and they are
printed to file "UL.dat " as $UAl=\psi^L_{\vk l}(i\omega,\vr)
U_{\vr\alpha}$ and "UR.dat" as $UAr=U^\dagger_{\beta\vr'}   \psi^R_{\vk l}(i\omega,\vr')$.

Finally, we need to solve for Bogolubov quasiparticles by
diagonalizing the following Hamiltonian
\begin{eqnarray}
H_{BG}=
\left(
\begin{array}{cc}
\varepsilon_{\vk,i\omega}-\mu & \bar{\Delta}^\vk(i\omega)\\
\bar{\Delta}^{\vk \dagger}(-i\omega) & -\varepsilon_{-\vk,-i\omega}^T+\mu
\end{array}
\right)
\end{eqnarray}
It turns out that $\varepsilon_{\vk,i\omega}$ and
$\bar{\Delta}(i\omega)$ do not have a branch-cut at $i\omega=0$ hence
we can safely take the zero frequency limit. We will also assume that
the lattice has inversion symmetry, hence
$\varepsilon_{-\vk}=\varepsilon_{\vk}$. We hence diagonalize
\begin{eqnarray}
H_{BG}=
\left(
\begin{array}{cc}
\varepsilon_{\vk,0}-\mu & \bar{\Delta}_\vk\\
\bar{\Delta}_{\vk}^{\dagger} & -\varepsilon_{\vk,0}^T+\mu
\end{array}
\right)
\end{eqnarray}
The difference between the smallest positive eigenvalue $\lambda^-$ and the
largest negative eigenvalue $\lambda^+$ is equal twice the gap $2\Delta=\lambda^+-\lambda^-$.

\end{document}
