%\documentclass[aps,showpacs,prb,floatfix,twocolumn]{revtex4}
\documentclass[aps,prb,floatfix,epsfig,singlecolumn,showpacs,preprintnumbers]{revtex4}
%%%%%%%%%%%%%%%%%%%%%%%%%%%%%%%%%%%%%%%%%%%%%%%%%%%%%%%%%%%%%%%%%%%%%%%%%%%%%%%%%%%%%%%%%%%%%%%%%%%%%%%%%%%%%%%%%%%%%%%%%%%%
\usepackage{amsmath,amssymb,graphicx,bm,epsfig}
\usepackage{color}
\usepackage{braket}

\newcommand{\eps}{\epsilon}
\newcommand{\vR}{{\mathbf{R}}}
\newcommand{\vF}{{\mathbf{F}}}
\renewcommand{\vr}{{\mathbf{r}}}
\newcommand{\hr}{{\hat{\textbf{r}}}}
\newcommand{\vk}{{\mathbf{k}}}
\newcommand{\vdelta}{{\mathbf{\delta}}}
\newcommand{\vK}{{\mathbf{K}}}
\newcommand{\vG}{{\mathbf{G}}}
\newcommand{\vq}{{\mathbf{q}}}
\newcommand{\vQ}{{\mathbf{Q}}}
\newcommand{\vPhi}{{\mathbf{\Phi}}}
\newcommand{\vS}{{\mathbf{S}}}
\newcommand{\cG}{{\cal G}}
\newcommand{\cR}{{\cal R}}
\newcommand{\cF}{{\cal F}}
\newcommand{\cT}{{\cal T}}
\newcommand{\cO}{{\cal O}}
\newcommand{\cH}{{\cal H}}
\newcommand{\cJ}{{\cal J}}
\newcommand{\cD}{{\cal D}}
\newcommand{\cU}{{\cal U}}
\newcommand{\cL}{{\cal L}}
\newcommand{\Tr}{\mathrm{Tr}}
\renewcommand{\a}{\alpha}
\renewcommand{\b}{\beta}
\newcommand{\g}{\gamma}
\renewcommand{\d}{\delta}
\newcommand{\npsi}{\underline{\psi}}
\renewcommand{\Im}{\textrm{Im}}
\renewcommand{\Re}{\textrm{Re}}
\newcommand{\cA}{{\cal A}}
\newcommand{\vcA}{\vec{\cal A}}
\newcommand{\vcB}{\vec{\cal B}}
\newcommand{\vcC}{\vec{\cal C}}
\newcommand{\cB}{{\cal B}}
\newcommand{\cC}{{\cal C}}
\usepackage{hyperref}

\begin{document}

\title{Some notes on the Free Energy in DFT+DMFT}
\author{Kristjan Haule}
\affiliation{Department of Physics, Rutgers University, Piscataway, NJ 08854, USA}
\date{\today}

%\begin{abstract}
%\end{abstract}
\pacs{71.27.+a,71.30.+h}
\date{\today}
\maketitle

\section{How is Free energy implemented?}

The details of the algorithm are published in 
\href{http://journals.aps.org/prl/abstract/10.1103/PhysRevLett.115.256402}{Phys. Rev. Lett. 115, 256402 (2015)}. 
Here we just mention some implementation details, which are not
completely detailed there.

The total energy is printed in column six of \textit{info.iterate}. We
compute it by
\begin{eqnarray}
E_6=E^H[\rho]+E^{XC}[\rho]+E_{nuc}-\Tr((V_H+V_{XC})\rho)+
\textcolor{red}{\Tr(\varepsilon^{DFT}_\vk \rho)}+
\textcolor{blue}{E_{imp}^{pot}-\Phi^{DC}[n_{imp}]}
\end{eqnarray}
The blue terms are computed from the impurity quantities, and the
black and red
are computed from the lattice quantities. The black are evaluated in \textit{lapw0} and \textit{lapwc}. The red
terms are evaluated in \textit{dmft2}.

The free energy is printed in column seven and eight of
\textit{info.iterate}. The seventh column is computed by
\begin{eqnarray}
F_7=E^H[\rho]+E^{XC}[\rho]+E_{nuc}-\Tr((V_H+V_{XC})\rho)
+\textcolor{red}{\Tr\log( G)+\mu N-\Tr\log(G_{loc})+\Tr((\varepsilon_{imp}+V_{DC})n_{loc})}
\\
+\textcolor{blue}{E_{imp}^{pot}
+\Tr\left((\Delta-\omega_n\frac{d\Delta}{d\omega_n})G_{imp}\right)
-\Phi^{DC}[n_{imp}]}
-\textcolor{green}{T S_{imp}}
\end{eqnarray}
The blue terms are computed from the impurity quantities, and the
black and red
are computed from the lattice quantities. The black are evaluated in \textit{lapw0} and \textit{lapwc}. The red
terms are evaluated in \textit{dmft2}. The green term is missing in
\textit{info.iterate} and needs to be computed at postprocessing as
explained in the above PRL.

The eight column is computed by
\begin{eqnarray}
F_8=E^H[\rho]+E^{XC}[\rho]+E_{nuc}-\Tr((V_H+V_{XC})\rho)
+\textcolor{red}{\Tr\log( G)+\mu N-\Tr((\Sigma-V_{DC})G_{loc})}
\\
+\textcolor{blue}{E_{imp}^{pot}
+\Tr\left((\Delta-\omega_n\frac{d\Delta}{d\omega_n})G_{imp}\right)
-\Tr\log(G_{imp})+\Tr(\varepsilon_{imp}n_{imp})+\Tr(\Sigma_{imp}G_{imp})
-\Phi^{DC}[n_{imp}]}
-\textcolor{green}{T S_{imp}}
\end{eqnarray}

The difference between the seventh and eight column is then
\begin{eqnarray}
F_8-F_7 = 
\textcolor{red}{\Tr\log(G_{loc})-\Tr(\Sigma G_{loc})-\Tr(\varepsilon_{imp}n_{loc})}
-\textcolor{blue}{\Tr\log(G_{imp})+\Tr(\Sigma G_{imp})+\Tr(\varepsilon_{imp}n_{imp})}
\end{eqnarray}

\end{document}