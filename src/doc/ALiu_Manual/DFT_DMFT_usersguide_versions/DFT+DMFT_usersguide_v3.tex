\documentclass[12 pt]{article}
\usepackage{graphicx}
% \graphicspath{{/home/aliu/Documents/Latex/DFT DMFT usersguide/DFT DMFT usersguide versions/}}
\usepackage{amsmath} % math formatting
\usepackage{amssymb} % math symbols
\usepackage{eulervm, bookman} % fonts for math and symbols
\usepackage[round]{natbib} % bibliography styles
\usepackage{textcomp}
\usepackage{lipsum} %
\usepackage[margin=1in, includefoot]{geometry}
\usepackage{fancyhdr} % change header
\pagestyle{fancy}
\fancyhead{}
\renewcommand{\headrulewidth}{0pt}


\title{
      {\Huge\textbf{User's Guide, DFT+DMFT\_14.1(2014)}}
      \_\_\_\_\_\_\_\_\_\_\_\_\_\_\_\_\_\_\_\_\_\_\_\_\_\_\_\_\_\_\_\_\_\_\_\_\_\_\_\_\_\_\_\_\_\_\_\_\_\_\_\_\\
      {\includegraphics[scale= 0.5]{DFT_DMFT}}
      \_\_\_\_\_\_\_\_\_\_\_\_\_\_\_\_\_\_\_\_\_\_\_\_\_\_\_\_\_\_\_\_\_\_\_\_\_\_\_\_\_\_\_\_\_\_\_\_\_\_\_\_\\
      } % adds title

\author{
      \large\textbf{Ang Liu, Peng Zhang and Ronald Cohen}\\
      Cohen's Theory Group\\
      5251 Broad Branch Road, N.W.\\
      Washington, D.C. 20015-1305\\
      United States of America\\
	} % adds author
\date{\today} % adds updated date
% use .PNG image format whenever possible
% save the image in same folder as . tex file

%\renewcommand{\baselinestretch}{2}
%\setlength{\baselineskip}{1cm}
\linespread{2}

\begin{document}

  \linespread{1.2}
  \maketitle
  \thispagestyle{empty} % suppress page number
  \newpage % add page before Contents
  \setcounter{page}{1} % reset page counter
  \pagenumbering{roman} % set page numbering to small roman numerals
  \renewcommand{\contentsname}{Table of Contents}

  \tableofcontents

  \newpage
  \pagenumbering{arabic} % change page numbering style from small roman to arabic
  \setcounter{page}{1}	% reset page number
  \part{\large\textbf{Introduction}}\label{part:intro}                                        
    \section{\textbf{Electronic structure of crystal and Density Functional Theory(DFT)}}\label{sec:DFT}


  Density-functional theory in the Kohn-Sham formulation is the basic tool for weakly interacting electronic 
systems and is widely used by the electronic structure community. We will review it using the effective action
 approach, which was introduced in this context by Fukuda.

  \emph{Choice of variables.} The density of electrons {$\rho {(\textbf{r})}$} is the central quantity of DFT
 and it is used as a physical variable in derivation of DFT functional.

  \emph{Construction of exact functional.} To construct the DFT functional, we probe the system with a time-
dependent source field \emph{J(x)}.

 This modifies the action of the system (9) as follows:

   $S\prime[J]$ = S + $\int $ {\emph {dx{J(x)$\psi^{+}(x)$ $\psi(x)$.}}

  The partition function Z becomes a functional of the auxiliary source field J,

%  Z[\emph{J}] = exp (-F[\emph{J}]) = $\int$ D[$\psi^\dag$ $\psi$]$e^{$-S^{\textquotesingle}[J]$}$.

  The effective action for the density, i.e., the density functional,is obtained as the Legendre transform of
 F with respect to $\rho (x)$.

  From this point forward, we restrict the source to be time independent because we will only be constructing 
the standard DFT. IF the time dependence were retained, one could formulate time-dependent density-functional
 theory. The density appears as the variational derivative of the free energy with respect to the source 
  \emph{The constraining field in DFT}. We demonstrate below that, in the context of DFT, the constraining 
field is the sume of the well-known exchange-correlation potential and the Hartree potential $V_{xc} + V_{H}$,
 and we refer to this quantity as $V_{int}$. This is the potential which must be added to the noninteracting
 Hamiltonian in order to yield the exact density of the full Hamiltonian. Mathematically, $V_{int}$ is a 
functional of the density which solves the equation

  The Kohn-Sham equation gives rise to a reference system of noninteracting particles, the so-called Kohn-Sham 
orbitals $\psi _{kj}$ which produce the interacting density

  Here the Kohn-Sham potential is $V_{KS}$ = $V_{ext}$ + $V_{int}$, are the Kohn-Sham energy bands and wave 
functions, \textbf{k} is a wave vector which runs over the first Brillouin zone, j is the band index, and 
is the Fermi function.

  \emph{Kohn-Sham Green's function}. Alternatively, the electron density can be obtained with the help of
 the Kohn-Sham Green's function, given by

   where $G_0$ is the noninteracting Green's function,

  and the density can then be computed from 

  The Kohn-Sham Green's function is defined in the entire space, where $V_{int}(\textbf{r})$ is adjusted 
such that the density of the system $\rho (\textbf{r})$ can be found from . It can also be expressed in 
terms of the Kohn-Sham particles in the following way:

  Kohn-Sham decomposition. Now we come to the problem of writing exact and approximate expressions for 
the functional. The strategy consists in performing an expansion of the functional in powers of electron
 charge. The Kohn-Sham decomposition consists of splitting the functional into the zeroth-order term and 
the remainder,

  This is equivalent to what Kohn and Sham did in their original work. In the first temr, $e^2 =0$ only 
for electron-electron interactions, and not for the interaction of the electron and the external potential. 
The first term consists of the kinetic energy of the Kohn-Sham particles and the external potential. The 
constraining field $\emph{J}_0$ is $V_{int}$ since it generates the term that needs to be added to the 
noninteracting action in order to get the exact density. Furthermore, functional integration of Eq. gives 
and from Eq. it follows that

  The remaining part $\Delta\Gamma_{DFT}(\rho)$ is the interaction energy functional which is decomposed 
into the Hartree and exchange-correlation energies,

  $\varPhi_{DFT}^{xc}[\rho]$ at zero temperature becomes the standard exchange correlation energy in DFT.

  Kohn-Sham equations as saddle-point equations. The density functional $\Gamma_{DFT}(\rho)$ can be regarded 
as a functional which is stationary in two variables $V_{int}$ and $\rho$. Extremization with respect to $V_{int}$ 
leads to Eq. , while stationary with respect to $\rho$ gives $V_{int}$=$\delta\Delta\Gamma/\delta\rho$, or 
equivalently

  where $V_{xc}(\textbf{r})$ is the exchange-correlation potential given by

  Equations and along with Eqs and or, equivalently, Eqs and , from the system of equations of the density-
functional theory. It should be noted that the Kohn-Sham equations give the true minimum of $\Gamma_{DFT}(\rho)$, 
and not only the saddle point, in contrast to spectral functional theories such as the BK method. 

  \emph{Exact representation for $\Phi_{DFT}^{xc}$.} The explicit form of the interaction functional 
$\Phi_{DFT}^{xc}[\rho]$ is not available. However, it may be defined by a power series expansion which can 
be constructed order by order using the inversion method. The latter can be given, albeit complicated, 
a diagrammatic interpretation. Alternatively, an expression involving integration by a coupling constant 
$\lambda e^2$ can be obtained using the Harris-Jones formula. One considers $\Phi_{DFT}[\rho, \lambda]$ at an 
arbitrary interaction $\lambda$ and expresses it as

  Here the first term is simply $K_{DFT}[G_{KS}]$ as given by Eq. which does not depend on $\lambda$. The second 
part is the unknown functional $\Phi_{DFT}^{xc}[\rho]$. The derivative with respect to the coupling constant in Eq. 
is given by the average

  where $\Pi_{\lambda}(x, x^{\prime})$ is the density-density correlation function at a given interaction strength 
$\lambda$ computed in the presence of a source which is $\lambda$ dependent and chosen so that the density of the 
system is $\rho$. Since , one obtains

  This expression has been used to construct more accurate exchange correlation functional. 

  \emph{Approximations. } Since is not known explicitly some approximations are needed. The LDA assumes,

  where is the exchange-correlation energy of the uniform electron gas, which is easily parametrized. $V_{eff}$ is 
given as an explicit function of the local density. In practice one frequently uses analytical formulas. The idea 
here is to fit a functional form to quantum Monte Carlo(QMC) calculations. Gradient corrections to the LDA have been 
worked out by Perdew and co-workers. They are also frequently used in DFT calculations. 

  \emph{Evaluation of the total energy.} At the saddle point, the density functional $\Gamma_{DFT}$ delievers the 
total free energy of the system,

  where the trace in the second term runs only over spatial coordinates and not over imaginary time. If temperature 
goes to zero, the entropy contribution vanishes and the total energy formula is recovered

  \emph{Assessment of the approach} From a conceptual point of view, the density-functional approach is radically 
different from the Green's-function theory (see Sec. below). The Kohn-Sham equations and describe the Kohn-Sham 
quasiparticles which are poles of $G_{KS}$ and are not rigorously identifiable with on-electron excitations. This 
is very different from the Dyson equation which determines the Green's function G, which has poles at the observable 
one-electron excitations. In principle the Kohn-Sham orbitals are a technical tool for generating the total energy. 
They are, however, not a necessary element of the approach as DFT can be formulated without introducing the Kohn-
Sham orbitals. In practice, they are an excellent first step in perturbative calculations of the one-electron Green's 
function in powers of screened Coulomb interaction, as, e.g., the \emph{GW} method. Both the LDA and GW methods are 
very successful in many materials in which one can apply the standard model of solids. However, in correlated electron 
systems this is not always the case. Our view is that this situation cannot be remedied by either using more complicated 
exchange-correlation functionals in density-functional theory or adding a finite number of diagrams in perturbation theory. 
As discussed above, the spectra of strongly correlated electron systems have both correlated quasiparticle bands and 
Hubbard bands which have no analog in one-electron theory.

  The density-functional theory can also be formulated for model Hamiltonians, the concept of density being replaced 
by the diagonal part of the density matrix in a site representation. It was tested in the context of the Hubbard 
model by , and . \citep {Kotliar2006}

    \newpage 
    \section{\textbf{Strongly correlated systems and\\ Dynamical Mean$\_$Field Theory(DMFT)}} \label{sec:DMFT}
      \subsection{Strongly correlated systems}\label{subsec:strongly correlated}

  What do we mean by a strongly correlated phenomenon? We can answer this question from the perspective of electronic 
structure theory, where one-electron excitations are well defined and represented as delta-function-like peaks showing 
the locations of quasiparticles at the energy scale of the electronic spectral functions. Strong correlations 
imply the breakdown of the effective one-particle description: the wave function of the system becomes essentially 
many-body-like, represented by combinations of Slater determinants, and the one-particle Green's functions no longer 
exhibit single-peaked features. 

  The development of methods for studying strongly correlated materials has a long history in condensed matter physics. 
The efforts of the many-body community have traditionally focused on the solution of model Hamiltonians (usually 
written for a given solid-state system on physical grounds) using techniques such as diagrammatic methods, quantum 
Monte Carlo simulations, exact diagonalization for finite-size clusters, density-matrix renormalization-group methods, 
and so on. The development of LDA + U and self-interaction corrected (SIC) methods, many-body perturbative approaches 
based on GW and its extensions, as well as the time-dependent version of the density-functional theory, have been 
carried out by the electronic structure community to address the problem of the strongly correlated materials. 
Some of the these techniques are already much more complicated and time consuming compared to the standard LDA-based 
algorithms, and therefore the real exploration of materials is frequently performed by simplified versions utilizing 
approximations such as the plasmon-pole form for the di-electric function, omitting the self-consistency within GW 
or assuming locality of the GW self-energy. 

  To motivate the dynamical mean-field theory approach we recall the nature of the one-electron (or one-particle) density 
of states of strongly correlated systems may display both renormalized quasiparticles and atomiclike states 
simultaneously. To describe this method one needs a technique which is able to treat quasiparticle bands and 
Hubbard bands on equal footing, and which is able to interpolate between atomic and band limits. Dynamical mean-
field theory is the simplest approach which captures these features; it has been extensively developed to study 
model Hamiltonians. Figure shows the development of the spectrum while increasing the strength of Coulomb interaction 
U as obtained by DMFT solution of the Hubbard model. It illustrates the necessity to go beyond static mean-field 
treatments in situations when the on-site Hubbard U becomes comparable with the bandwidth W. 

  Model Hamiltonian based DMFT methods have successfully described regimes $U\diagup W \gtrsim 1$.However, to describe
 strongly correlated materials we need to incroporate realistic electronic structure calculations. The low-temperature 
physics of systems near localization-delocalization crossover is nonuniversal, system specific, and sensitive to the 
lattice structure and orbital degeneracy which is unique to each compound. We believe that incroporating this 
information into the many-body treatment of this system is a neccessary first step before more general lessons about 
strong-correlation phenomena can be drawn. In this respect, we recall that DFT in its common approximations, such 
as LDA or GGA, brings a system specific description into calculations. Despite the great success of DFT for studying 
weakly correlated solids, it has not been able thus far to address strongly correlated phenomena. So, we see that 
both density-functional-based and many-body model Hamiltonian approaches are to a large extent complementary to 
each other and hence can be merged. One-electron Hamiltonians, which are necessarily generated within density-functional 
approaches (i.e., the hopping terms), can be used as input for more challenging many-body calucations. This path 
was undertaken by Anisimov et al. who introduced the LDA+DMFT method of electronic structure for strongly correlated 
systems and applied it to the photoemmision spectrum of . Near the Mott transition, this system shows a number of 
features incompatible with the one-electron description. The electronic structure of Fe has been shown to be in 
better agreement with experiment within DMFT in comparison with LDA. The photoemission spectrum near the Mott transition 
in has been studied, as well as issues connected to the finite-temperature magnetism of Fe and Ni were explored. 

  Despite these successful developments, we also emphasize a more ambitious goal: to build a general method which treats 
all bands and all electrons on the same footing, determines both hoppings and interactions internally using a fully 
self-consistent procedure, and accesses both energetics and spectra of correlated materials. These efforts have been 
undertaken in a series of papers which gave us a functional description of the problem in analogy to the density-
functional theory, and its self-consistent implementation is illustrated on pluonium. 

  To summarize, there are two roads in approaching the problem of simulating correlated materials properties, which 
we illustrate in Fig. Dynamical mean-field theory has been useful in both instances. To describe these efforts in a 
language understandable by both elctronic structure and many-body communities, and to stress qualitative differences 
and great similarities between DMFT and LDA, we start our review with a general many-body framework based on the 
effective action approach to strongly correlated systems. 

  \citep {Kotliar2006}

      \newpage
      \subsection{DMFT formulism}\label{subsec:DMFT formulism}

DMFT is a mapping of a lattice problem onto an impurity problem. Therefore at the heart of every DMFT calculation 
is the solution of the Anderson impurity model.

  The SDFT should be viewed as an exact theory whose manifestly local constraining field is an auxiliary mass 
operator introduced to reproduce the local part of the Green's function of the system, exactly like the Kohn-Sham 
potential is an auxiliary operator introduced to reproduce the density of electrons in DFT. However, to obtain
 practial results, we need practical approximations. The dynamical mean-field thoery can be thought of as an approximation 
to the exact SDFT functional in the same spirit as LDA appears as an approximation to the exact DFT functional.

  The diagrammatic rules of the exact SDFT functional can be developed but they are more complicated than in 
Baym-Kadanoff theory as discussed by . The single-site DMFT approximation to this functional consists of taking 
to be a sum of all graphs, constructed with as a vertex and as a propagator, which are two-particle irreducible, 
namely, . This together with Eq. defines the DMFT approximation to the exact spectral density functional. 

  It is possible to arrive at this functional by summing up diagrams or using the coupling constant integration 
trick with a coupling-dependent Green's function having the DMFT form, namely, with a local self-energy. This 
results in

  with in Eq. the self-energy of the Anderson impurity model. It is useful to have a formulation of this DMFT 
funcational as a function of three variables, namely, combining the hybridization with that atomic Green's 
function to form the Weiss function, one can obtain the DMFT equations from the stationary point of a functional 
of , and the Weiss field:

  One can eliminate and from Eq. using stationary conditions and recover a functional of the Weiss field function 
only. This form of the functional, applied to the Hubbard model, allowed the analytical determination of the nature 
of the transition and the characterization of the zero-temperature critical points. Alternatively eliminating and 
in favor of one obtains the DMFT approximation to the self-energy functional discussed in. 

      \newpage
    \section{\textbf{DFT+DMFT}}\label{sec:DFT+DMFT}
      \subsection{Basics of DFT + DMFT}\label{subsec:Basics of DFT+DMFT}

  As stressed in this review, the ultimate goal of our research is a fully first-principles electronic structure 
method which can treat strongly correlated systems (i.e., see). Because this ambitious methodology is still 
under development, we continue to rely on the simplified approach which is DFT+DMFT. One of the great merits of 
DFT+DMFT is that it is a nearly first-principles methos. The user only needs to input the structure, the atomic 
species, and the interactions (i.e., U). The DFT+DMFT code suite is broken into three codes. 

  The first part is the DFT code, which is simply a modified version of LMTART. It has nearly identical input 
files, with minor differences in how correlated orbitals are specified. Therefore the main inputs of this code 
are the unit cell and the atomic species. The main role of this code is to generate and export the converged DFT 
Hamiltonian matrix in a local basis for each k point. Therefore this code essentially generates the parameters of 
the unperturbed Hamiltonian automatically. This information is needed to construct the local Green's function. 

  The second part is the code which implements the DMFT self-consistency condition, which requires a choice of 
correlated orbital and double counting. This code takes the Hamiltonian matrix and the self-energy as input, and 
provides the bath function as output.

  The third part is the various codes which solve the Anderson impurity model, and have been descibed in the 
first section. These codes take the bath function as input and provide the self-energy, which is used in the 
sel-consistency condition in the preceding step.

  These three pieces allow one to perform a non-self-consistent DFT+DMFT calculation as follows. First, the DFT 
code is used to generate the local, orthogonalized Hamiltonian matrix at each k point. Second, one starts with 
a guess for the self-energy and uses the DMFT self-consistency condition code to find the bath function. Third, 
the bath function is fed into the impurity solver producing a new self-energy. The second and third steps are 
then repeated until DMFT self-consistency is achieved. This is considered a non-self-consistent DFT+DMFT calculation. 
This process should be continued until both the total density and the local Green's function have converged. 

  One should note that the above pieces which compose the DFT+DMFT suite are three separate codes. Therefore one 
must write a simple script to iterate the above algorithm until self-consistency is reached (i.e., the self-energy 
converges to within some tolerance). Additionally, the DFT portion of this code suite (i.e., the first part) can 
in principle be replaced by any DFT code as long as a local basis set is generated. 

      \newpage
      \subsection{Flow charts}\label{subsec:Flow charts}

\smallskip
\begin{figure}
 \centering
 \includegraphics[scale=0.45]{dftpart}
 \caption{DFT loop}
 \label{DFT loop figure}
\end{figure}

\begin{figure}
 \centering
 \includegraphics[scale=0.45]{dmftpart}
 \caption{DMFT loop}
 \label{DMFT loop figure}
\end{figure}

\begin{figure}
 \centering
 \includegraphics[scale=0.6]{dmft1}
 \caption{DMFT1}
 \label{DMFT1 figure}
\end{figure}

\begin{figure}
 \centering
 \includegraphics[scale=0.6]{dmft2}
 \caption{DMFT2}
 \label{DMFT2 figure}
\end{figure}





  \newpage
  \part{\large\textbf{Installation}}\label{part:installation}

  \newpage    
  \setcounter{section}{0}
  \part{\large\textbf{Examples}}\label{part:examples}
    \section{\textbf{Self-consistent cycles in DFT+DMFT for FeO}}\label{sec:self-consistent cycles}
      \subsection{DFT}\label{subsec:DFT}
	\subsubsection{LAPW0}\label{subsubsec:LAPW0}
	\subsubsection{LAPW1}\label{subsubsec:LAPW1}
	\subsubsection{LAPWS0}\label{subsubsec:LAPWS0}
      \subsection{DMFT}\label{subsec:DMFT}
	\subsubsection{x\_dmft.py dmft1}\label{subsubsec:dmft1}
	\subsubsection{Impurity solver}\label{subsubsec:impurity solver}
	\subsubsection{x\_dmft.py dmft2}\label{subsubsec:dmft2}
      \subsection{Core charge and charge mixing}\label{subsec:core charge}
	\subsubsection{x core}\label{subsubsec:x core}
	\subsubsection{x mixer}\label{subsubsec:x mixer}
      \subsection{Density of States(DOS), Spectra and Fermi Surface(FS)}\label{subsec:DOS}
      \subsection{Transport properties}\label{subsec:transport}
      \subsection{Magnetic ordering}\label{subsec:magnetic}

  \newpage
  \setcounter{section}{0} % reset section number
  \appendix 
      
      \section{Programming guide for DFT+DMFT}\label{sec:programming}
      \section{Trouble shooting}\label{sec:trouble shooting}
      \section{List of options}\label{sec:list of options}
      \section{Frequently Asked Questions}\label{sec:FAQ}
      \section{Glossaries and Indexes}\label{sec:indexes}
      \section{Frequently used terms}\label{sec:FUT}
      \section{Bibliograph}\label{sec:Bibliograph}

\bibliographystyle{plainnat} % set bibliography style
\bibliography{DFT+DMFT_usersguide_v3} % include bibliography files


  \clearpage % add page after contents

\end{document}

