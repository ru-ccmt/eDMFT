%\documentclass[aps,showpacs,prb,floatfix,twocolumn]{revtex4}
\documentclass[aps,prb,floatfix,epsfig,twocolumn,showpacs,preprintnumbers]{revtex4}
%%%%%%%%%%%%%%%%%%%%%%%%%%%%%%%%%%%%%%%%%%%%%%%%%%%%%%%%%%%%%%%%%%%%%%%%%%%%%%%%%%%%%%%%%%%%%%%%%%%%%%%%%%%%%%%%%%%%%%%%%%%%
\usepackage{amsmath,amssymb,graphicx,bm,epsfig}
\usepackage{color}

\newcommand{\eps}{\epsilon}
\newcommand{\vR}{{\mathbf{R}}}
\renewcommand{\vr}{{\mathbf{r}}}
\newcommand{\vt}{{\mathbf{t}}}
\newcommand{\vG}{{\mathbf{G}}}
\newcommand{\hr}{{\hat{\textbf{r}}}}
\newcommand{\vk}{{\mathbf{k}}}
\newcommand{\vK}{{\mathbf{K}}}
\newcommand{\vq}{{\mathbf{q}}}
\newcommand{\vQ}{{\mathbf{Q}}}
\newcommand{\vPhi}{{\mathbf{\Phi}}}
\newcommand{\vS}{{\mathbf{S}}}
\newcommand{\cG}{{\cal G}}
\newcommand{\cF}{{\cal F}}
\newcommand{\cT}{{\cal T}}
\newcommand{\cH}{{\cal H}}
\newcommand{\cJ}{{\cal J}}
\newcommand{\cD}{{\cal D}}
\newcommand{\cL}{{\cal L}}
\newcommand{\Tr}{\mathrm{Tr}}
\renewcommand{\a}{\alpha}
\renewcommand{\b}{\beta}
\newcommand{\g}{\gamma}
\renewcommand{\d}{\delta}
\newcommand{\npsi}{\underline{\psi}}
\renewcommand{\Im}{\textrm{Im}}
\renewcommand{\Re}{\textrm{Re}}
\newcommand{\cA}{{\cal A}}



\begin{document}

\title{Rotations in LAPW}
\maketitle


%\section{Definitions}

An element of the space group $\Gamma$ is composed of a rotation $R$
and translation $\vt$. How does the symmetry operation $\Gamma$ act on
$\vr$ is clear. But how does it work on wave functions
$\chi_{\vk+\vK}(\vr)\equiv |\vk+\vK\rangle \equiv |\vG\rangle$? In our
notation, the operation acts in the following way
\begin{eqnarray}
&& \langle \vr |\Gamma\;|\vG\rangle =   \langle R^{-1}\vr +\vt|\vG\rangle =
e^{i\vG \vt}\langle \vr|R\vG\rangle\\
&& \langle \vr |\Gamma^{-1}|\vG\rangle =   \langle R(\vr -\vt)|\vG\rangle =
e^{-i R^{-1} \vG \vt}\langle \vr|R^{-1}\vG\rangle
\end{eqnarray}

To compute the electronic charge, or partial density of state, or DMFT
local green's function, we need the following matrix elements
\begin{equation}
\langle l,m| i\vk\rangle \equiv  \int Y^*_{l,m}(\vr) \psi_{i\vk}(\vr) d\vr
\end{equation}

The problem is that $\vk$ is usually given only in the irreducible
Brillowuin zone ($\vk_{IBZ}$). Furthermore the projectors should be
specified in a local coordinate system attached to a specific atom,
which is different from the global coordinate system. The following
series of operations is typically performed in the electronic
structure codes
\begin{eqnarray}
\langle l,m| R_0\; T_{\tau_1}\;\Gamma_n\;
\Gamma_\alpha|i\vk_{IBZ}\rangle
\label{E4}
\end{eqnarray}
Here
\begin{itemize}
\item $R_0$ is a local rotation in which $Y_{l,m}(\vr)$ is specified. This
  transofrmation can be specified by user. Typically user
  enters a new local $z$-axis and a new local $x$-axis.
  In Wien2K code, this transformation is named \textit{crotloc(:,:)}.
\item $T_{\tau_1}$ shifts the origin of the coordinate system from the
  origin to the local coordinate system attached to an atom. Here we
  shift only to the first atom of certain type.
  In Wien2K code, this shift is named $POS(:,lfirst)$.
\item $\Gamma_n$ is the transformation from the first atom of a certain
  type to all equivalent atoms of this type. It is composed of a
  rotation $R_n$ and a translation $\tau_n$, named 
  \textit{rotij}(:,:,latom) and 
  \textit{tauij(:,latom)} in Wien2K.
\item $\Gamma_\alpha$ transforms an irreducible $\vk$ point to a
  reducible $\vk$-point in the same star.
  It is composed of a rotation $R_{\alpha}$ and translation
  $\tau_{\alpha}$, named \textit{tmat(:,:,$\gamma$)} and
  \textit{tau(:,$\gamma$)}.
\end{itemize}

Our transformation Eq.~(\ref{E4}) is composed of coefficients
$\langle \vG |\vk_{IBZ}\rangle$, which are equal for all points in a
star, and do not depend on local coordinate system. The other part of
the coefficents depends on symmetry operations and local rotation
\begin{eqnarray}
\sum_\vG \langle l,m|R_0\; T_{\tau_1}\; \Gamma_n\;\Gamma_\alpha|\vG\rangle\langle \vG |\vk_{IBZ}\rangle
\end{eqnarray}
Finally we need to evaluate
\begin{eqnarray}
\langle l,m|R_0\; T_{\tau_1}\; \Gamma_n\;\Gamma_\alpha|\vG\rangle=\\
\exp\left({i\vG\tau_\alpha + i (R_\alpha \vG) \tau_n}\right)\langle
l,m|R_0\; T_{\tau_1}|R_n R_\alpha \vG\rangle=\\
\exp\left({ i\vG\tau_\alpha + i (R_\alpha \vG) \tau_n + i(R_n R_\alpha \vG)\tau_1}\right)\langle
l,m|R_0 R_n R_\alpha \vG\rangle
\end{eqnarray}

In Wien2K, we hence need to add the following phase factor
\begin{eqnarray}
G(:) tau(:,\gamma)+tmat(:,:,\gamma)G(:)tauij(:,latom)\\
+rotij(:,:,latom)tmat(:,:,\gamma)G(:)POS(:,lfirst)
\end{eqnarray}
The wave vector $\vG$ needs to be rotated by the following sequence of
transformations
\begin{eqnarray}
 crotloc(:,:) rotij(:,:,latom) tmat(:,:,\gamma)G(:)
\end{eqnarray}

\end{document}
