%\documentclass[aps,showpacs,prb,floatfix,twocolumn]{revtex4}
\documentclass[aps,prb,floatfix,epsfig,twocolumn,showpacs,preprintnumbers]{revtex4}
%%%%%%%%%%%%%%%%%%%%%%%%%%%%%%%%%%%%%%%%%%%%%%%%%%%%%%%%%%%%%%%%%%%%%%%%%%%%%%%%%%%%%%%%%%%%%%%%%%%%%%%%%%%%%%%%%%%%%%%%%%%%
\usepackage{amsmath,amssymb,graphicx,bm,epsfig}
\usepackage{color}

\newcommand{\eps}{\epsilon}
\newcommand{\vR}{{\mathbf{R}}}
\renewcommand{\vr}{{\mathbf{r}}}
\newcommand{\hr}{{\hat{\textbf{r}}}}
\newcommand{\vk}{{\mathbf{k}}}
\newcommand{\vK}{{\mathbf{K}}}
\newcommand{\vq}{{\mathbf{q}}}
\newcommand{\vQ}{{\mathbf{Q}}}
\newcommand{\vPhi}{{\mathbf{\Phi}}}
\newcommand{\vS}{{\mathbf{S}}}
\newcommand{\cG}{{\cal G}}
\newcommand{\cF}{{\cal F}}
\newcommand{\cT}{{\cal T}}
\newcommand{\cH}{{\cal H}}
\newcommand{\cJ}{{\cal J}}
\newcommand{\cD}{{\cal D}}
\newcommand{\cL}{{\cal L}}
\newcommand{\Tr}{\mathrm{Tr}}
\renewcommand{\a}{\alpha}
\renewcommand{\b}{\beta}
\newcommand{\g}{\gamma}
\renewcommand{\d}{\delta}
\newcommand{\npsi}{\underline{\psi}}
\renewcommand{\Im}{\textrm{Im}}
\renewcommand{\Re}{\textrm{Re}}
\newcommand{\cA}{{\cal A}}



\begin{document}

\title{DMFT-Wien2k Manual}
\author{Kristjan Haule}
\affiliation{Department of Physics, Rutgers University, Piscataway, NJ 08854, USA}
\date{\today}

%\begin{abstract}
%\end{abstract}
\pacs{71.27.+a,71.30.+h}
\date{\today}
\maketitle


\section{Instalation}

The program consists of many independent programs, which are written in
C++, fortran90, and Python.

Here are some important steps to instal the package
\begin{itemize}
\item Edit the file \verb Makefile.in  to set the path to compilers,
  compiler options, and
  libraries on your system. You will need
  \begin{itemize}
    \item intel mkl library
    \item intel fortran compiler
    \item gnu C++ compiler
    \item Python with numpy and scipy
    \item Python CXX package
  \end{itemize}
\item type \verb make %
\item set an environment variable \verb WIEN_DMFT_ROOT  %
  to point to the directory you plan to install the code
\item type \verb make  \verb install %
\end{itemize}


\begin{small}
Note: When installing numpy on a linux distribution, make sure to have gcc in the
environment variable CC=gcc and CXX=g++, because otherwise linux
will install f2py compiled with icc and icpc, which does NOT work at all
(the options sent to icc are wrong at compilation time of f2py).
\end{small}

\section{Short intro to running DMFT-Wien2k code}

The package consists of many independent modules, which are written in
C++, fortran90, and Python.

The highest level scripts are written in Python (".py" files). You can
always get help on python script by running the script with the
argument "-h" or "$--$help".


The most important steps of the LDA+DMFT executions are:
\begin{itemize}
\item \textbf{initialization}:
  After LDA is converged with the Wien2K package, one should run\\
  \verb >  \verb init_dmft.py  \\%
  and follow the instructions.
  The script will create two input files: $case$.indmfl and
  $case$.indmfi .
  The first contains information of how a local self-energy can be
  added to the Kohn-Sham potential. The second connects the local
  self-energy with the output self-energy of the impurity solvers.
  
\item \textbf{preparation}: Prepare additional input files
  \begin{itemize}
  \item \verb params.dat  \\%
    The file must contains information for the impurity solver, number of
    self-consistent steps, etc.
  \item \verb sig.inp  \\%
    Contains starting guess for the input self-energy. Here zero
    dynamic self-energy is usualy a good starting point. This can be
    generated by \\
    \verb >  \verb szero.py  -e Edc  \\%
    "Edc" should be a number close to $U(n-1/2)$, where $n$ is
    expected impurity occupation. This creates a good guess for
    double-couting and $\Sigma(\infty)$, but ultimately a good guess
    for the impurity levels.

    The frequency mesh for the self-energy is very important. It
    is generated in the following way:\\
    If you have a good self-energy from some other run, you can copy it to
    the working directory, and the script will take the mesh from
    current \textit{sig.inp}. If \textit{sig.inp} does not exist, you
    should specify the following arguments to the \textit{szero.py} script
    \begin{itemize}
    \item -n int : Number of frequency points
    \item -T float : Temperature for the imaginary axis mesh
    \item -L float : cuttof energy on the real axis
    \end{itemize}
    The real-axis mesh generated in this way is not efficient, and one
    should rather generatea more efficient "tan" mesh with alternative
    script, and copy it to \textit{sig.inp}.
    
  \item \verb Sigma.000  \\%
    This file is needed only on real axis. It is a guess for the
    self-energy of pseudoparticles. If you have no experience in
    generating this file, you should take it from some example
    run. You only need the first column, which gives a frequency for
    the self-energy.
    
  \end{itemize}
\item \textbf{self-consistent run}: By invoking\\
  \verb >  \verb run_dmft.py  \\%
  the python script will produce self-consistent LDA+DMFT solution.

  Useful informtation is stored in  the following log files:\\
  \verb dmft_info.out   -- top-most information about the LDA+DMFT run\\
  $case$\verb .dayfile   -- list of all executed steps and current convergence.\\
  \verb dmft1_info.out  -- information about the dmft1 step\\
  $case$\verb .outputdmf1   -- more information on dmft1 step from  fortran routines.\\
  \verb dmft2_info.out  -- information about the dmft2 step\\
  $case$\verb .outputdmf2  -- more information on dmft2 step from  fortram routines.\\

\end{itemize}

  The self-consistent calculation, performed by
  (\verb run_dmft.py )
  can also be performed by a few steps, which can be invoked
  sequentially by the user.
  These steps are
  \begin{itemize}
  \item \textit{LDA potential} : Is computed by \\
    \verb >  \verb x   \verb lapw0  %
  \item \textit{LDA eigensystem} : Is computed by \\
    \verb >  \verb x   \verb lapw1  %
  \item \textit{so-coupling} : When needed, so-coupling ss added by \\
    \verb >  \verb x   \verb lapwso  %

  \item self-energy split: Is invoked by \\
    \verb >  \verb ssplit.py  %
    
    The self-enery for all atoms and all orbitals is stored in
    \verb sig.inp . The first two lines contain the double-couting $E_{dc}$,
    and $\Sigma(\infty)$. The columns correspond to the dynamic part of
 the self-energy. Each correlated block $(atom,l)$ needs an
 independent input file in the \verb dmft1  and \verb dmft2  step. Even
 if two atoms are equivalent, they need independent input
 self-energy. For each such block, a file \verb sig.inp[r]  is
 generated (where \verb [r] is positive integer ), which contains
 $\Sigma(\omega)-E_{dc}$ for each correlated block.
    
    
  \item \textbf{dmft1 step} : Can be invoked by \\
    \verb >  \verb x_dmft.py  \verb dmft1  %

    Computes the local Green's function and the
    hybridization function.
    The ouput files are\\
    $case$\verb .cdos  -- density of states, \\%
    $case$\verb .gc[r]  -- local green's functios, \\%
    $case$\verb .dlt[r]  -- hybridization function, \\%
    $case$\verb .Eimp[r]  -- impurity levels, \\%
    $case$\verb .outputdmf1  -- logging information, \\%

  \item Prepare impurity hybridization: Invoked by \\
    \verb >  \verb sjoin.py  \verb -m  mixing-parameter \\
    It  produces hybridization function for all impurity
    problems. 
    \verb dmft1  step produces
    hybridization function and impurity levels for all correlated
    blocks ($case$.\verb .dlt[r] ).
    In DMFT, the number of impurity problems can be smaller
    than the number of correlated atoms (either because some atoms are
    equivalentm, or some atoms are grouped together in clusters). From
    the hybridization functions
    ($case$.\verb .dlt[r] , $case$.\verb .Eimp[r] ),
    the impurity hybridization
    (\verb imp.[r]/Delta.imp , \verb imp.[r]/Eimp.inp  ) are
    generated.
    
    
  \item \textit{\textbf{Impurity solver}} : Solves the auxiliary impurity problem.
    Currently supported impurity solvers are:
    \begin{itemize}
      \item CTQMC  -- continuous time quantum Monte Carlo
      \item OCA    -- the one crossing approximation
      \item NCA    -- To invoke it, the mode should be "OCA", but the executable should be "nca".
    \end{itemize}
    
  \item  combine impurity self-energies : Invoked by \\
    \verb >  \verb sgather.py  \\%
    It  take the impurity self-energy, and creates a
    common file with self-energy.
    The impurity solvers produce the new self-energy in \verb imp.[r]/sig.out . %
    The result is combined into a single file named \verb sig.inp  .
    
  \item self-energy split: Invoked by\\
    \verb >  \verb ssplit.py  \\%
    In the next step (dmft2) we want to use the new self-energy just
    produced by the impurity solver. Hence, we create
    \verb sig.inp[r]  again from single self-energy file \verb sig.inp , %
     just created.
  
  \item \textbf{dmft2 step} : Recomputes the electronic charge using
    LDA+DMFT self-energy.
    The output is stored in \\
    $case$\verb .clmval   -- the new valence charge density\\
    \verb EF.dat        -- the new chemical potential\\
    $case$\verb .cdos3    -- occupied part of the DOS ( just for  debuging purposes. )\\
    \verb dmft2_info.out   -- some logging information\\
    $case$ \verb .outputdmf2  -- more logging information from the fortram subroutines.\\
    $case$ \verb scf2       -- more information from the fortran.\\

  \item \textbf{lcore} : Computes LDA core density by invoking \\
    \verb >  \verb x   \verb lcore  %
  \item \textbf{mixer} : Produces the total electronic charge, and
    mixes it with previous density by broyden-like method. Invoked by: \\
    \verb >  \verb x   \verb mixer  %
  
  \end{itemize}
  



\begin{thebibliography}{99}

\end{thebibliography}

\end{document}
    outputdmf1
  \item sjoin.py -m mixing-parameter
    
  \item \textit{\textbf{Impurity solver}} : Solves the auxiliary impurity problem

  \item sgather.py
  \item ssplit.py
  
  \item \textbf{dmft2} : Recomputes the electronic charge using    LDA+DMFT self-energy

  \item lcore
  \item mixer
  
  \end{itemize}
  



The two DMFT steps (dmft1 and dmft2) can be invoked by the following
command
\begin{verbatim}
> x_dmft.py dmft2
\end{verbatim}
The two executables


\begin{thebibliography}{99}

\end{thebibliography}

\end{document}
