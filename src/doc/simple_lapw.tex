%\documentclass[aps,showpacs,prb,floatfix,twocolumn]{revtex4}
\documentclass[aps,prb,floatfix,epsfig,twocolumn,showpacs,preprintnumbers]{revtex4}
%%%%%%%%%%%%%%%%%%%%%%%%%%%%%%%%%%%%%%%%%%%%%%%%%%%%%%%%%%%%%%%%%%%%%%%%%%%%%%%%%%%%%%%%%%%%%%%%%%%%%%%%%%%%%%%%%%%%%%%%%%%%
\usepackage{amsmath,amssymb,graphicx,bm,epsfig}
\usepackage{color}

\newcommand{\eps}{\epsilon}
\newcommand{\vR}{{\mathbf{R}}}
\renewcommand{\vr}{{\mathbf{r}}}
\newcommand{\hr}{{\hat{\textbf{r}}}}
\newcommand{\vk}{{\mathbf{k}}}
\newcommand{\vdelta}{{\mathbf{\delta}}}
\newcommand{\vK}{{\mathbf{K}}}
\newcommand{\vq}{{\mathbf{q}}}
\newcommand{\vQ}{{\mathbf{Q}}}
\newcommand{\vPhi}{{\mathbf{\Phi}}}
\newcommand{\vS}{{\mathbf{S}}}
\newcommand{\cG}{{\cal G}}
\newcommand{\cF}{{\cal F}}
\newcommand{\cT}{{\cal T}}
\newcommand{\cH}{{\cal H}}
\newcommand{\cJ}{{\cal J}}
\newcommand{\cD}{{\cal D}}
\newcommand{\cU}{{\cal U}}
\newcommand{\cL}{{\cal L}}
\newcommand{\Tr}{\mathrm{Tr}}
\renewcommand{\a}{\alpha}
\renewcommand{\b}{\beta}
\newcommand{\g}{\gamma}
\renewcommand{\d}{\delta}
\newcommand{\npsi}{\underline{\psi}}
\renewcommand{\Im}{\textrm{Im}}
\renewcommand{\Re}{\textrm{Re}}
\newcommand{\cA}{{\cal A}}



\begin{document}

\title{Some notes on LAPW}
\author{Kristjan Haule}
\affiliation{Department of Physics, Rutgers University, Piscataway, NJ 08854, USA}
\date{\today}

%\begin{abstract}
%\end{abstract}
\pacs{71.27.+a,71.30.+h}
\date{\today}
\maketitle

\begin{widetext}
The LAPW basis takes the form:
\begin{eqnarray}
&& \chi_{\vk+\vK}(\vr) = \frac{1}{\sqrt{V}} e^{i(\vk+\vK)\vr}
=\frac{4\pi i^l}{\sqrt{V}}e^{i(\vk+\vK)\vr_\alpha}Y_{lm}^*(R(\hat{\vk}+\hat{\vK}))j_l(|\vk+\vK||\vr-\vr_\alpha|)Y_{lm}(R(\vr-\vr_\alpha))
  \qquad interstitial\\
&& \chi_{\vk+\vK}(\vr) = \left( a_{lm} u_l(|\vr-\vr_\alpha|) + b_{lm}
  \dot{u}_l(|\vr-\vr_\alpha|)\right)
  Y_{lm}(R(\hat{\vr}-\hat{\vr_\alpha}))\qquad MT-sphere
\end{eqnarray}
The matching condition at the MT-sphere $S$ gives
\begin{eqnarray}
\left(
\begin{array}{cc}
u_l(S) & \dot{u}_l(S)\\
\frac{d}{dr} u_l(S) & \frac{d}{dr} \dot{u}_l(S)
\end{array}
\right)
\left(
\begin{array}{c}
a_{lm}\\
b_{lm}
\end{array}
\right)=
\frac{4\pi i^l}{\sqrt{V}}e^{i(\vk+\vK)\vr_\alpha}Y_{lm}^*(R(\hat{\vk}+\hat{\vK}))
\left(
\begin{array}{c}
j_l(|\vk+\vK|S)\\
\frac{d}{dr} j_l(|\vk+\vK|S)
\end{array}
\right)
\end{eqnarray}
with the solution
\begin{eqnarray}
\left(
\begin{array}{c}
a_{lm}\\
b_{lm}
\end{array}
\right)=
\frac{4\pi i^l}{\sqrt{V}}e^{i(\vk+\vK)\vr_\alpha}Y_{lm}^*(R(\hat{\vk}+\hat{\vK}))
%
\left(
\begin{array}{cc}
\frac{d}{dr} \dot{u}_l(S)  & -\dot{u}_l(S)\\
-\frac{d}{dr} u_l(S) & u_l(S)       
\end{array}
\right)
\frac{1}{u_l(S) \frac{d}{dr} \dot{u}_l(S)-\dot{u}_l(S) \frac{d}{dr} u_l(S)}
%
\left(
\begin{array}{c}
j_l(|\vk+\vK|S)\\
\frac{d}{dr} j_l(|\vk+\vK|S)
\end{array}
\right)
\end{eqnarray}

The two solutions satisfy the following equations
\begin{eqnarray}
&&\left(-\frac{d^2}{dr^2}+\frac{l(l+1)}{r^2}+V_{KS}(r)-\varepsilon \right) r u_l(r)=0\\
&&\left(-\frac{d^2}{dr^2}+\frac{l(l+1)}{r^2}+V_{KS}(r)-\varepsilon \right)r \dot{u}_l(r)= r u_l(r)
\end{eqnarray}
We multiply the first equation by $r\dot{u}_l(r)$ and the second by $ru_l(r)$ to obtain
\begin{eqnarray}
\int_0^S dr\left\{ r\dot{u}_l(r) \left(-\frac{d^2}{dr^2}\right) r u_l(r)
-r u_l(r)
\left(-\frac{d^2}{dr^2}\right)r \dot{u}_l(r)\right\}=-\int_0^S dr r^2 u_l(r)u_l(r)
\end{eqnarray}
Integration by parts gives
\begin{eqnarray}
\left[
- r\dot{u}_l(r) \frac{d}{dr} \left(r u_l(r)\right)
+r u_l(r) \frac{d}{dr} \left(r \dot{u}_l(r)\right)\right]_0^S
=-1
\end{eqnarray}
which finally leads to
\begin{eqnarray}
\dot{u}_l(S) \frac{d}{dr} u_l(S)
-u_l(S) \frac{d}{dr} \dot{u}_l(S) =\frac{1}{S^2}
\end{eqnarray}

We can than simplify the solution for $a_{lm}$ and $b_{lm}$ to 
\begin{eqnarray}
\left(
\begin{array}{c}
a_{lm}\\
b_{lm}
\end{array}
\right)=
\frac{4\pi i^l}{S^2 \sqrt{V}}e^{i(\vk+\vK)\vr_\alpha}Y_{lm}^*(R(\hat{\vk}+\hat{\vK}))
%
\left(
\begin{array}{c}
\dot{u}_l(S)\frac{d}{dr} j_l(|\vk+\vK|S)-\frac{d}{dr} \dot{u}_l(S) j_l(|\vk+\vK|S)\\
\frac{d}{dr} u_l(S) j_l(|\vk+\vK|S)- u_l(S) \frac{d}{dr} j_l(|\vk+\vK|S)
\end{array}
\right)
\end{eqnarray}
This equation is implemented in Wien2k, and also in both dmft1 and
dmft2 steps.


To compute the projector, we need the overlap between a localized function
$\phi(r) Y_L(\vr)$, and Kohn-Sham states
\begin{eqnarray}
\cU^{\vk,\vr_\alpha}_{i,m}=\langle \phi_l Y_{lm}|\psi_{\vk i}\rangle  =
\sum_\vK C^\vk_{i\vK} \langle \phi_l(|\vr-\vr_\alpha|) Y_{lm}(R(\hat{\vr}-\hat{\vr}_\alpha))|\chi_{\vk+\vK}(\vr)\rangle
\end{eqnarray}
If function $\phi(r)$ extends sufficiently outside its MT-sphere, the
overlap $\cU^{\vk,\vr_\alpha}_{i,m}$ will have non-zero contribution
from all other MT-spheres. However, we will use only the envelope
function outside its center sphere, because the increased charge in
the neighboring spheres really should not be counted here as charge
contribution to $\phi(r)$ function.

Therefore we have  only two contributions. Inside MT-sphere we have
\begin{eqnarray}
\langle \phi_l(|\vr-\vr_\alpha|) Y_{lm}(\hat{\vr}-\hat{\vr}_\alpha)|\chi_{\vk+\vK}(\vr)\rangle  =
\sum_\kappa a^\kappa_{lm} \int_0^S \phi(r)  u_l^\kappa(r) r^2 dr
\end{eqnarray}
and outside MT-sphere we get
\begin{eqnarray}
\langle \phi_l(|\vr-\vr_\alpha|) Y_{lm}(R(\hat{\vr}-\hat{\vr}_\alpha))|\chi_{\vk+\vK}(\vr)\rangle  =
\frac{4\pi  i^l}{\sqrt{V}}  e^{i(\vk+\vK)\vr_\alpha}Y_{lm}^*(R(\hat{\vk}+\hat{\vK}))
\int_S^{S_2} \phi_l(r) j_l(|\vk+\vK|r)r^2 dr
\end{eqnarray}


\section{Free energy and Total Energy}


The equation for the total energy is
\begin{eqnarray}
E = \Tr(H_0 G) + \frac{1}{2}\Tr(\Sigma G) - \Phi^{DC}[n_{loc}] + \Phi^H[\rho]+\Phi^{xc}[\rho]
\end{eqnarray}
where
$$H_0 = -\nabla^2 + \delta(\vr-\vr')V_{ext}(\vr)$$
We typically evaluate it in the following way
\begin{eqnarray}
E = \Tr((-\nabla^2+V_{ext}+V_{H}+V_{xc}) G) + \frac{1}{2}\Tr(\Sigma G)-\Phi^{DC}[\rho_{loc}] 
-\Tr((V_H+V_{xc})\rho) + \Phi^H[\rho]+\Phi^{xc}[\rho]
\end{eqnarray}
Namely, we use the Green's function of the solid to evaluate:
\begin{eqnarray}
E_1 = \Tr((-\nabla^2+V_{ext}+V_{H}+V_{xc}) G) -\Tr((V_H+V_{xc})\rho) + \Phi^H[\rho]+\Phi^{xc}[\rho]
\end{eqnarray}
and the impurity to evaluate
\begin{eqnarray}
E_2 =  \frac{1}{2}\Tr(\Sigma_{imp} G_{imp})-\Phi^{DC}[\rho_{imp}] 
\end{eqnarray}
Notice that $\frac{1}{2}\Tr(\Sigma_{imp} G_{imp})$ is not evaluated as
a Matsubara sum, but we rather compute it from probabilities of atomic
states, i.e.,
\begin{eqnarray}
 \frac{1}{2}\Tr(\Sigma_{imp} G_{imp}) = \sum_m P_m E_m - \sum_\alpha
 \varepsilon_{imp}^\alpha n_{imp}^\alpha
\end{eqnarray}



The free energy functional is
\begin{eqnarray}
\Gamma[G] = \Tr\log G-\Tr\log((G_0^{-1}-G^{-1})G) + \Phi^{H}[\rho]+\Phi^{xc}[\rho]+\Phi^{DMFT}[G_{loc}]-\Phi^{DC}[\rho_{loc}]
\end{eqnarray}
hence stationarity gives
\begin{eqnarray}
G^{-1}-G_0^{-1} + V_H + V_{xc}+\Sigma_{DMFT}-V_{dc}=0
\end{eqnarray}
and hence
\begin{eqnarray}
F = \Tr\log G
-\Tr(\Sigma G) +
\Tr(V_{dc}\rho_{loc}) + \Phi^{DMFT}[G_{loc}]-\Phi^{DC}[\rho_{loc}]
-\Tr((V_H + V_{xc})\rho)+
\Phi^{H}[\rho]+\Phi^{xc}[\rho]
\end{eqnarray}
Since $F_{imp}$ contains $\Phi^{DMFT}$, i.e,
\begin{eqnarray}
F_{imp} = \Tr\log G_{imp} - \Tr(\Sigma_{imp} G_{imp}) + \Phi^{DMFT}[G_{imp}]  
\end{eqnarray}
we can write
\begin{eqnarray}
F = \Tr\log(G)-\Tr\log(G_{loc})+F_{imp}+\Tr(V_{dc} \rho_{loc}) -
\Phi^{DC}[\rho_{loc}]
-\Tr((V_H + V_{xc})\rho)+
\Phi^{H}[\rho]+\Phi^{xc}[\rho]  
\end{eqnarray}
where
$$F_{imp} = E_{imp}-T S_{imp}$$
and
$$E_{imp}=
\Tr((\Delta+\varepsilon_{imp}-\omega_n\frac{\partial\Delta}{\partial\omega_n})
G_{imp}) + \frac{1}{2}\Tr(\Sigma_{imp} G_{imp}) -T S_{imp}
$$
Hence
\begin{eqnarray}
F+T S_{imp} = \Tr\log(G)-\Tr\log(G_{loc})+
%
\Tr((\Delta+\varepsilon_{imp}-\omega_n\frac{\partial\Delta}{\partial\omega_n})
G_{imp}) + \frac{1}{2}\Tr(\Sigma_{imp} G_{imp})
%
+\Tr(V_{dc} \rho_{loc}) -
\Phi^{DC}[\rho_{loc}]\nonumber\\
-\Tr((V_H + V_{xc})\rho)+
\Phi^{H}[\rho]+\Phi^{xc}[\rho]  
\end{eqnarray}
which can also be cast into the form
\begin{eqnarray}
F+T S_{imp} = \Tr\log(G)-\Tr\log(G_{loc})+
\Tr((\Delta-\omega_n\frac{\partial\Delta}{\partial\omega_n}+\varepsilon_{imp}+V_{dc})G_{loc})
+ \frac{1}{2}\Tr(\Sigma_{imp} G_{imp})
-\Phi^{DC}[\rho_{imp}]\nonumber\\
-\Tr((V_H + V_{xc})\rho)+
\Phi^{H}[\rho]+\Phi^{xc}[\rho]  
\end{eqnarray}
We thus compute the following quantities with the Green's function of
the solid:
\begin{eqnarray}
F_1 = \Tr\log(G)-\Tr\log(G_{loc})+
\Tr((\Delta-\omega_n\frac{\partial\Delta}{\partial\omega_n}+\varepsilon_{imp}+V_{dc})G_{loc})
-\Tr((V_H + V_{xc})\rho)+
\Phi^{H}[\rho]+\Phi^{xc}[\rho]    
\end{eqnarray}
and the following with the impurity:
\begin{eqnarray}
F_2 =   \frac{1}{2}\Tr(\Sigma_{imp} G_{imp})-\Phi^{DC}[\rho_{imp}] -T S_{imp}
\end{eqnarray}
Notice that $F_2$ is similar to $E_2$ (except for the entropy term), hence $E_{solid}$ and
$F_{solid}$ contain exactly the same Monte Carlo noise.

\end{widetext}

\end{document}
