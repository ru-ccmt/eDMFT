%\documentclass[aps,showpacs,prb,floatfix,twocolumn]{revtex4}
\documentclass[aps,prb,floatfix,epsfig,twocolumn,showpacs,preprintnumbers]{revtex4}
%%%%%%%%%%%%%%%%%%%%%%%%%%%%%%%%%%%%%%%%%%%%%%%%%%%%%%%%%%%%%%%%%%%%%%%%%%%%%%%%%%%%%%%%%%%%%%%%%%%%%%%%%%%%%%%%%%%%%%%%%%%%
\usepackage{amsmath,amssymb,graphicx,bm,epsfig}
\usepackage{color}

\newcommand{\eps}{\epsilon}
\newcommand{\vR}{{\mathbf{R}}}
\renewcommand{\vr}{{\mathbf{r}}}
\newcommand{\hr}{{\hat{\textbf{r}}}}
\newcommand{\vk}{{\mathbf{k}}}
\newcommand{\vdelta}{{\mathbf{\delta}}}
\newcommand{\vK}{{\mathbf{K}}}
\newcommand{\vq}{{\mathbf{q}}}
\newcommand{\vQ}{{\mathbf{Q}}}
\newcommand{\vPhi}{{\mathbf{\Phi}}}
\newcommand{\vS}{{\mathbf{S}}}
\newcommand{\cG}{{\cal G}}
\newcommand{\cF}{{\cal F}}
\newcommand{\cT}{{\cal T}}
\newcommand{\cH}{{\cal H}}
\newcommand{\cJ}{{\cal J}}
\newcommand{\cD}{{\cal D}}
\newcommand{\cL}{{\cal L}}
\newcommand{\Tr}{\mathrm{Tr}}
\renewcommand{\a}{\alpha}
\renewcommand{\b}{\beta}
\newcommand{\g}{\gamma}
\renewcommand{\d}{\delta}
\newcommand{\npsi}{\underline{\psi}}
\renewcommand{\Im}{\textrm{Im}}
\renewcommand{\Re}{\textrm{Re}}
\newcommand{\cA}{{\cal A}}



\begin{document}

\title{How to plot magnetic calculation in non-magnetic BZ}
\author{Kristjan Haule}
\affiliation{Department of Physics, Rutgers University, Piscataway, NJ 08854, USA}
\date{\today}

%\begin{abstract}
%\end{abstract}
\pacs{71.27.+a,71.30.+h}
\date{\today}
\maketitle

\section{Derivation 1}

The general expression for the GS is
\begin{eqnarray}
G_{\vk_0}(\omega) = \sum_i\int d\vr d\vr' e^{-i\vk_0 \vr}\psi_{i\vk}(\vr)\frac{1}{\omega-\epsilon_\vk^i}\psi_{i\vk}^*(\vr')e^{i\vk_0\vr'}
\end{eqnarray}
with $k_0$ an arbitrary momentum. Here 
\begin{equation}
\psi_{i\vk}(\vr)=\sum_{i\vK} A_{i\vK}^{\vk}\chi_{\vk+\vK}(\vr)
\end{equation}
are Kohn-Sham solutions, and $\chi$ are basis functions.
We hence have
\begin{widetext}
\begin{eqnarray}
G_{\vk_0}(\omega) = \sum_i\sum_{\vK_1,\vK_2} \langle e^{-i\vk_0
  \vr}|\chi_{\vk+\vK_1}(\vr)\rangle
A_{i\vK_1}^\vk\frac{1}{\omega-\epsilon_\vk^i}A_{i\vK_2}^{\vk *}\langle
\chi_{\vk+\vK_2}(\vr')|e^{i\vk_0\vr'}\rangle
\end{eqnarray}
\end{widetext}

For plane wave basis, the matrix elements are
$$\langle e^{-i\vk_0 \vr}|\chi_{\vk+\vK}(\vr)\rangle = \delta_L(\vk+\vK-\vk_0) $$
where $\delta_L$ requires that $\vk+\vK=\vk_0$ up to reciprocal vector
of the non-magnetic unit cell!

In order to avoid computing annoying matrix elements, we will use the
same expression also in the LAPW basis set. We just need to
generalized it for the non-orthogonal basis set.
The generalization is
\begin{widetext}
\begin{eqnarray}
G_{\vk_0}(\omega)=\sum_{\vK_1,\vK_2,i}\delta_L(\vk+\vK_1-\vk_0)A_{i\vK_1}^\vk\frac{1}{\omega-\epsilon_\vk^i}A_{i\vK_2}^{\vk *} \delta_L(\vk+\vK_2-\vk_0)\langle\chi_{\vk+\vK_1}|\chi_{\vk+\vK_2}\rangle
\end{eqnarray}
In order to plot fat-bands (with character), we can express $G$ inside
the muffin thin sphere in the following way
\begin{eqnarray}
G_{\vk_0}^{L_1  L_2}(\omega)=\sum_{\vK_1\vK_2,i}\delta_L(\vk+\vK_1-\vk_0)A_{i\vK_1}^\vk
a_{L_1}^{\kappa_1}(\vk+\vK_1)\frac{1}{\omega-\epsilon_\vk^i}A_{i\vK_2}^{\vk  *} a_{L_2}^{\kappa_2 *}(\vk+\vK_2)\delta_L(\vk+\vK_2-\vk_0)\langle u_{l_1}^{\kappa_1}|u_{l_2}^{\kappa_2}\rangle
\end{eqnarray}
which can also be written as
\begin{eqnarray}
G_{\vk_0}^{L_1  L_2}(\omega)=\sum_{i\kappa_1\kappa_2}
{\cal A}_{i L_1}^{\kappa_1}(\vk)
\frac{1}{\omega-\epsilon_\vk^i}
{\cal A}_{i L_2}^{\kappa_2 *}(\vk)
\langle u_{l_1}^{\kappa_1}|u_{l_2}^{\kappa_2}\rangle
\end{eqnarray}
\end{widetext}
with
\begin{equation}
{\cal A}_{i L}^{\kappa}(\vk)=\sum_{\vK}\delta_L(\vk+\vK-\vk_0)A_{i\vK}^\vk a_{L}^{\kappa}(\vk+\vK)
\end{equation}

This expression is used to compute partial density of states in QTL
and DMFT, except that $\delta_L$-functions then requires that
$\vk=\vk_0$ and $\vK$ can be any reciprocal vector.

For magnetic calculation, we need to perform calculation in bigger
unit cell. Hence we have shorter reciprocal vectors. Out of reciprocal
vectors of the magnetic BZ, we need to find those which are reciprocal
vectors of non-magnetic BZ. Then the sum over $\vK$ above should be
performed only over the non-magnetic reciprocal vectors.

%If $\vk_0$ is
%in first-BZ of the magnetic unit cell, or is outside it, a different
%set of  reciprocal vectors will be summed over.

\section{Alternative derivation}

The green's function of LDA+DMFT in real space, expressed in terms of
the Kohn-Sham states $\psi_{i\vk}(\vr)$, is
\begin{equation}
 G(\vr,\vr') = \sum_{ij\vk}\psi_{i\vk}(\vr)\left(\frac{1}{\omega+\mu-\varepsilon_\vk-P_\vk\Sigma}\right)_{ij}\psi_{j\vk}^*(\vr')
\end{equation}
Below, we will use the notation
\begin{equation}
  \left(\frac{1}{\omega+\mu-\varepsilon_\vk-P_\vk\Sigma}\right)_{ij}\equiv  g_{\vk ij}
\end{equation}


Magnetic unit cell is bigger, and hence we can write Green's function
on sublatice A, on sublatice B, and the off-diagonal Green's function.
If the Green's function is written in the position representation
within the unit cell, it is easy to derive the green's function in the
non-magnetic unit cell. The result is
\begin{eqnarray}
\left(
\begin{array}{cc}
 G_{\vk,AA} & G_{\vk,AB}\\
 G_{\vk,BA} & G_{\vk,BB}
\end{array}
\right)
\rightarrow\\
G_{\vk} = G_{\vk,AA} + G_{\vk,BB} + G_{\vk,AB} e^{i\vk\vdelta} + G_{\vk,BA}e^{-i\vk\vdelta}
\end{eqnarray}
Here $\delta$ is the vector connecting sublatice $A$ and sublatice
$B$. For the checkerboard AFM state, this vector is for example $(1,0,0)$. 

We want to write the LAPW Green's function in terms of the four
components of sublattices. We first note that the Kohn-Sham solutions $\psi_{i\vk}(\vr)$
are expanded in terms of LAPW basis set functions
$\chi_{\vk+\vK}(\vr)$ in the following way
\begin{equation}
\psi_{i\vk}(\vr) = \sum A^\vk_{i\vK}\chi_{\vk-\vK}(\vr)
\end{equation}
where $A^\vk_{\vk-\vK}$ are eigenvectors, written in \textit{case.vector}.
Since the LAPW basis functions transform
under shift in the same way as plane waves, we have
\begin{widetext}
\begin{eqnarray}
 G(\vr_1+\vdelta_1,\vr_2+\vdelta_2) = 
 \sum_{ij\vk,\vK,\vK'}
 A^\vk_{i\vK} e^{i(\vk-\vK)\vdelta_1}\chi_{\vk-\vK}(\vr_1) g_{\vk ij}
 \chi^*_{\vk-\vK'}(\vr_2)e^{-i(\vk-\vK')\vdelta_2} A^{\vk *}_{j\vK'}
\end{eqnarray}
The generalized expression for $G$ in the non-magnetic unit cell, in terms of the four components of
the Green's function, is
\begin{equation}
G_\vk = \int d\vr \left[ G(\vr,\vr) + G(\vr,\vr+\delta)e^{i\vk\delta}+G(\vr+\delta,\vr)e^{-i\vk\vdelta}+G(\vr+\delta,\vr+\delta)\right]  
\end{equation}
Using the above expression for $G(\vr_1+\vdelta1,\vr_2+\vdelta_2)$, we
have
\begin{eqnarray}
 G_\vk = 
 \sum_{ij\vK,\vK'}
 A^\vk_{i\vK} O^\vk_{\vK\vK'}A^{\vk *}_{j\vK'} g_{\vk ij}
(1+e^{i\vK'\vdelta}+e^{-i\vK\vdelta}+e^{i(\vK'-\vK)\vdelta}) 
\end{eqnarray}
\end{widetext}
where overlap is
\begin{equation}
O^\vk_{\vK\vK'} = \langle \chi_{\vk-\vK'}|\chi_{\vk-\vK}\rangle.
\end{equation}
It is clear that $e^{i \vK\vdelta}$ is unity for the reciprocal
vectors of the non-magnetic unit cell, and it is $-1$ for the new
reciprocal vectors, which are not part of the non-magnetic unit
cell. Hence, if one of the $\vK$ in the above expression is the "new"
reciprocal vector, and one is "old" (part of the non-magnetic
reciprocal set), the sum of the exponents in the bracket vanishes. The
same is true when both $\vK$ and $\vK'$ are "new" vectors, because
$\vK'-\vK$ is now the "old" vector, and hence $e^{i(\vK'-vK)\vdelta}=1$.
We see that only the terms which include the original non-magnetic
reciprocal vectors survive in the above expression. Hence the
expression is equivalent to the expression in the previous chapter,
but it is more convenient for implementation.
To compute it, we only
need Kohn-Sham eigenvectors $A^\vk_{i\vK}$, since the overlap can also
be expressed in terms of the overlap. The eigenvalue problem demands
$A O A^\dagger  = 1$, hence
\begin{equation}
\sum_{\vK\vK'} A^\vk_{i\vK} O^\vk_{\vK \vK'} A^\vk_{j\vK'} = \delta_{ij}
\end{equation}
We use SVD decomposition to invert eigenvectors (because they are not
quadratic matrice). The SVD is
\begin{equation}
A = U Z V  
\end{equation}
where $U$ and $V$ are unitary matrices, and $Z$ are singular
values. We then have
\begin{eqnarray}
O = V^\dagger \frac{1}{Z} U^\dagger U \frac{1}{Z} V.
\end{eqnarray}

We would like to write the final expression in a more compact way. We
define the "coherence"-like factors
\begin{equation}
 C^\vk_{ji} = 
 \sum_{\vK \vK'} A^\vk_{i\vK} O^\vk_{\vK\vK'}A^{\vk *}_{j\vK'}(1+e^{i\vK'\vdelta}+e^{-i\vK\vdelta}+e^{i(\vK'-\vK)\vdelta}) 
\end{equation}
and note that $G$ is then simply given by
\begin{equation}
 G_\vk =  \sum_{ij} C^\vk_{ji}\; g_{\vk ij}.
\end{equation}

Finally, it is convenient to express the Green's function in terms of
the LDA+DMFT eigenvalues and LDA+DMFT eigenvectors. We note that the
LDA+DMFT green's function expressed in Kohn-Sham basis is
\begin{equation}
g_{\vk ij} = A^{R}_{\vk\omega,i l} \frac{1}{\omega+\mu-\varepsilon_{\vk
    l\omega}} A^{L}_{\vk\omega,l j}
\end{equation}
Hence, we can define the LDA+DMFT frequency dependent "coherence" factors
\begin{eqnarray}
\widetilde{C}^\vk_{l\omega} = \sum_{ij} A^{L}_{\vk\omega,l j} C^\vk_{ji} A^{R}_{\vk\omega,i l} 
\end{eqnarray}
In terms of these "coherence" factors we finally have
\begin{equation}
G_\vk = \sum_l 
\frac{\widetilde{C}^\vk_{l\omega}}{\omega+\mu-\varepsilon_{\vk\omega l}}
\end{equation}

\end{document}
